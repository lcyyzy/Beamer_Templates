\input{beamer_head}


% 使用 \part,\section,\subsection 等命令组织文档结构
% 使用 \frame 命令制作幻灯片

\begin{document}

\logo{\includegraphics[height=0.09\textwidth]{fudan2.png}}
\title[课程项目报告]{基于区域分割的阴影检测和去除算法研究}
\author[刘超颖-14CS]{刘超颖-14CS}
\institute[FDU]{计算机科学与技术学院}
\date{\today}

% 定义目录页
\AtBeginPart{
  \frame{
    \frametitle{\partpage}
    \begin{multicols}{2}
% 如果目录过长,可以打开这个选项分两栏显示
      \tableofcontents
% 使用这个命令自动生成目录
    \end{multicols}
  }  
}

% 在每个Section前都会加入的Frame
\AtBeginSection[]
{
  \begin{frame}<beamer>
    \frametitle{}
	\setcounter{tocdepth}{2}
    \tableofcontents[currentsection,currentsubsection]
  \end{frame}
}
% 在每个Subsection前都会加入的Frame
\AtBeginSubsection[]
{
  \begin{frame}<beamer>
%\begin{frame}<handout:0>
% handout:0 表示只在手稿中出现
    \frametitle{Outline}
	\setcounter{tocdepth}{2}
    \tableofcontents[currentsection,currentsubsection]
% 显示在目录中加亮的当前章节
  \end{frame}
}

\begin{frame}
  \titlepage
\end{frame}

\begin{frame}[plain]
  \frametitle{Table of Contents}
  \setcounter{tocdepth}{2}
  \tableofcontents
\end{frame}


\section{引言}
\begin{frame}
\frametitle{算法的整体思路}
\begin{itemize}
	\setlength\itemsep{1cm}
	\item 传统的全局优化算法难以处理阴影区域内部材质发生变化的情况:
\end{itemize}
\begin{figure}
\center
%\includegraphics[scale=0.2]{1}
\end{figure}

\end{frame}
\begin{frame}
\frametitle{算法的整体思路}
\begin{itemize}
	\item 考虑将图像分割成一块块的小区域;
	\item 以区域为单位进行阴影检测和去除;
	\item 充分利用区域和区域之间是否光照条件相同,是否为相同材料这类信息。
\end{itemize}
\begin{figure}
\begin{minipage}[t]{0.5\linewidth}
\centering
%\includegraphics[width=2.2in]{2}
\end{minipage}%
\begin{minipage}[t]{0.5\linewidth}
\centering
%\includegraphics[width=2.2in]{3}
\end{minipage}
\end{figure}
\end{frame}

\section{阴影检测}
\begin{frame}
\frametitle{算法核心思路}
\begin{itemize}
	\setlength\itemsep{0.5cm}
	\item 对每个区域提特征,判断是否为阴影区;
	\item 区域和区域之间的关系能否为我们判断阴影区域提供帮助?
	\pause
	\begin{itemize}
		\item 相同材料,相同亮度
		\item 相同材料,不同亮度
		\item 不同材料
	\end{itemize}
	\item 构造能量函数,求解最优取值方案.
\begin{align}
\widehat{y}= & \arg \max_y \sum_{i=1}c_i^{shadow}y_i+\alpha_1\sum_{\{i,j\}\in E_{diff}}c_{ij}^{diff}(y_i-y_j) \\
&  -\alpha_2\sum_{\{i,j\}\in E_{same}}c_{ij}^{same}[y_i\ne y_j]
\end{align}
\end{itemize}
\end{frame}
\section{阴影去除}
\begin{frame}
\frametitle{算法核心思路}
\begin{itemize}
	\item 传统方法中阴影区域直接用整个非阴影区域的颜色进行补偿;
	\item 我们将图像分割成区域后,可以单独考虑每块阴影区域的颜色;
	\item 如果一块阴影区与一块非阴影区具有相同的材质,则按照这块非阴影区的颜色去修改阴影区比较合适;
	\item CIEL*a*b空间直方图均衡;
\end{itemize}
\begin{figure}
\center
%\includegraphics[scale=0.3]{10}
\end{figure}
\end{frame}
\begin{frame}
\frametitle{算法的改进}
\begin{itemize}
	\item 原算法对边缘的处理不够充分:
	\begin{itemize}
		\item 存在一些半影区的像素没有被标记为阴影像素;
		\item 在非阴影区内部可能存在小于一半的阴影像素;
	\end{itemize}
\end{itemize}
\begin{figure}
\center
%\includegraphics[scale=0.3]{4}
\end{figure}
\end{frame}
\begin{frame}
\frametitle{算法的改进}
\begin{itemize}
	\item 使用形态学方法找出这些可能是阴影或者未处理的阴影像素;
	\item 用in-painting算法对这些区域进行填充;
\end{itemize}
\begin{figure}
\center
%\includegraphics[scale=0.3]{5}
\end{figure}
\end{frame}
\section{实验}
\begin{frame}
\frametitle{算法的实现}
\begin{itemize}
	\item 阴影检测和阴影去除的核心代码均用Matlab实现;
	\item 比较费时的过程用C++实现再封装为Matlab借口,比如SAP最大流算法;
	\item 为了进行对比实现了一种传统的全局处理的方法;
\end{itemize}
\end{frame}
\begin{frame}
\frametitle{阴影检测算法}
\begin{itemize}
	\item 使用了三个数据集进行测试,评价方法采用逐像素比较;
\end{itemize}
\begin{figure}
\center
%\includegraphics[scale=0.4]{6}
\end{figure}
\end{frame}
\begin{frame}
\frametitle{阴影去除算法}
\begin{itemize}
	\item 使用RMSE评价方法:
\end{itemize}
\begin{figure}
\center
%\includegraphics[scale=0.5]{9}
\end{figure}
\end{frame}
\begin{frame}
\frametitle{实际运行效果}
%\includegraphics[scale=0.45]{7}
\end{frame}
\begin{frame}
\frametitle{实际运行效果}
\begin{figure}
\begin{minipage}[t]{0.5\linewidth}
\centering
%\includegraphics[width=2.2in]{11}
\end{minipage}%
\begin{minipage}[t]{0.5\linewidth}
\centering
%\includegraphics[width=2.2in]{12}
\end{minipage}
\end{figure}
\end{frame}
\begin{frame}
	\begin{spacing}{1.5}
	\begin{center}
	\Huge{\textbf{Thanks for your attention!}}

	\Huge{\textit{Any questions?}}
\end{center}
\end{spacing}
\end{frame}
\end{document}
